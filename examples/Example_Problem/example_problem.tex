\documentclass[14pt]{article}

\usepackage[margin=0.5in,top=0.5in,bottom=0.5in]{geometry}

\begin{document}

{\begin{center}
{\bf\large  Physics 3020 -- Methods of Computational Physics -- 
Fall 2015}\\[2mm]
Example Problem \\[2mm]
\end{center}

A very simple model of population growth is given by the model of
Robert May, George Oster and Jim Yorke, which states that the
population of a $k$-th generation is given by
\begin{equation} \label{law}
N^k = \lambda N^{k-1} ( 1 - N^{k-1}).
\end{equation}
Here $N^k$ can take values between 0 and 1, $0 \leq N^k \leq 1$, and
provides some measure of the size of a given generation.  As long as a
population is sufficiently small, the size of the next generation
should be proportional to the size of the previous generation.  That's
taken care of by the term $N^{k-1}$ in (\ref{law}).  When the size of
the population reaches the maximum of what can possibly be supported,
$N=1$ in this model, over-population should lead to a decrease in the
size of the next generation.  That's accomplished by the term ($1 -
N^{k-1}$) in (\ref{law}).  The factor $\lambda$ is a constant of
proportionality, the size of which may lead to completely different
behavior of the population development.  In particular,
\begin{itemize}
\item  $\lambda <\lambda_1$ leads to extinction (i.e.~$N^k \rightarrow 0$ as
$k \rightarrow \infty$),
\item for $\lambda_1 < \lambda < \lambda_2$ 
the population settles down to a constant value (greater than
zero),
\item for $\lambda_2 < \lambda < \lambda_3$ the population goes
through repeating cycles, 
\item and for $\lambda > \lambda_3$ we find {\em
chaotic} behavior (at least for some values for $\lambda$).  
\end{itemize}
Chaotic behavior means that an arbitrarily small
difference between two initial values $N^0$ leads to an arbitrarily
large difference between the populations after a finite number of
steps. 

\begin{enumerate}
\item
What is the maximum value that $\lambda$ may take to insure 
that $N^k \leq 1$?

\item
Write a computer program that
computes population size sequences $N^k$.  To search for chaotic
behavior, you may want to evolve two sequences simultaniously,
starting one with $N_0$, say, and the other with $N_0(1 + \delta)$,
where $\delta \ll 1$ is a small number.

\item
Run your program from
part (b) for various values of $\lambda$ (it's ok to always start with
the same starting value $N_0$, for example $N_0 = 1/2$) and describe
the different kinds of behavior that you find.  Locate the transition
from one behavior to another by ``bracketing'' (i.e.~find two very
similar values of $\lambda$ so that the smaller value leads to one
behavior and the larger value to another) and determine $\lambda_1$,
$\lambda_2$, and $\lambda_3$.

\end{enumerate}



\end{document}
